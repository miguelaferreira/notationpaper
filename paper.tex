\RequirePackage{fixltx2e}
\documentclass[runningheads,a4paper]{llncs}

\usepackage[utf8]{inputenc}

\usepackage{tikz}
\usetikzlibrary{arrows,matrix}
\usetikzlibrary{backgrounds}

\usepackage[american]{babel}

\usepackage{graphicx}

%extended enumerate, such as \begin{compactenum}
\usepackage{paralist}

%put figures inside a text
%\usepackage{picins}
%use
%\piccaptioninside
%\piccaption{...}
%\parpic[r]{\includegraphics ...}
%Text...

%Sorts the citations in the brackets
%\usepackage{cite}

%for easy quotations: \enquote{text}
\usepackage{csquotes}

\usepackage[T1]{fontenc}

%better font, similar to the default springer font
\usepackage{lmodern}
%if more space is needed, exchange lmodern by mathptmx
%\usepackage{mathptmx}

%enable margin kerning
\usepackage{microtype}

%for demonstration purposes only
\usepackage[math]{blindtext}

%unobstrusive usage of hyperref
\ifnum\pdfoutput>0
\usepackage[
%pdfauthor={},
%pdfsubject={},
%pdftitle={},
%pdfkeywords={},
bookmarks=false,
breaklinks=true,
colorlinks=true,
linkcolor=black,
citecolor=black,
urlcolor=black,
%pdfstartpage=19,
pdfpagelayout=SinglePage
]{hyperref}
%enables correct jumping to figures when referencing
\usepackage[all]{hypcap}
\else
\usepackage{hyperref}
\fi

\usepackage[capitalise,nameinlink]{cleveref}
%Nice formats for \cref
\crefname{section}{Sect.}{Sect.}
\Crefname{section}{Section}{Sections}
\crefname{figure}{Fig.}{Fig.}
\Crefname{figure}{Figure}{Figures}

\usepackage{xspace}
%\newcommand{\eg}{e.\,g.\xspace}
%\newcommand{\ie}{i.\,e.\xspace}
\newcommand{\eg}{e.\,g.,\ }
\newcommand{\ie}{i.\,e.,\ }

% correct bad hyphenation here
\hyphenation{op-tical net-works semi-conduc-tor}

\begin{document}

%Works on MiKTeX only
%hint by http://goemonx.blogspot.de/2012/01/pdflatex-ligaturen-und-copynpaste.html
%This allows a copy'n'paste of the text from the paper
\input glyphtounicode.tex
\pdfgentounicode=1

%\title{Paper Title}
\title{Putting Thought into Notation}
%\title{Relations with Forks and Handles}
%If Title is too long, use \titlerunning
%\titlerunning{Short Title}

%Single insitute
%\author{Firstname Lastname \and Firstname Lastname}
%If there are too many authors, use \authorrunning
%\authorrunning{First Author et al.}
%\institute{...}

%Multiple insitutes
%Currently disabled
%
%\iffalse
%Multiple institutes are typeset as follows:
\author{Joost Visser\inst{1}\inst{2} \and Miguel Ferreira\inst{3} }
%If there are too many authors, use \authorrunning
%\authorrunning{First Author et al.}

\institute{
Software Improvement Group\\
Amsterdam\\
\email{j.visser@sig.eu}\and
Radboud University\\
Nijmegen\and
Schuberg Philis\\
Schiphol-Rijk\\
\email{mferreira@schubergphilis.com}
}
%\fi

\maketitle

\begin{abstract}
The work of José Nuno Oliveira is elegant. He strives relentlessly for simplicity, conciseness, symmetry, and generality. This quest for elegance finds expression in his efforts to carefully choose or craft notation. Writing things down differently allows thinking about them more clearly and reasoning about them more effectively. In this paper, we revisit a cornerstone of José Nuno’s work, the relational calculus, and examine how a novel notation for relations could bring out their basic properties more clearly. Down with arrows and harpoons that hinder our understanding. Let’s fit handles and forks onto relations to disclaim partiality and confusion.
\end{abstract}

\keywords{...}

%%%%%%%%%%%%%%%%%%%%%%%%%%%%%%%%%%%%%%%%%%%%%%%%%%%%%%%%%%%%%%%%%%%%%%%%%%%%%%%
\section{Introduction}\label{sec:intro}
%%%%%%%%%%%%%%%%%%%%%%%%%%%%%%%%%%%%%%%%%%%%%%%%%%%%%%%%%%%%%%%%%%%%%%%%%%%%%%%

\begin{itemize}
\item Provide an example of where JNO argues for better notation.

Related extracts from \cite{DBLP:journals/tse/OliveiraF13}:
\begin{itemize}
  \item[p305, c2] makes the point that traditional engineering disciplines make use of higher abstraction notations to lead with difficult problems (e.g. sets of equations that get collapsed into a single matricial equation), and that this ability reveals maturity of the respective fields.
  \item[p305, c2] \it{… notation scales up and quantity does not disturb quality.}
  \item[p307, c2] \it{Arrow notation makes it possible to express relational formulae using diagrams. This is a major ingredient of the method because it provides a graphical way of picturing relation types and relation constraints.}
  \item[p321, c1] \it{… binary relations are naturally pictured as arrows in diagrams. In this way, not only types and operations (data types) but also constraints (business rules) can be displayed by semantically rich drawings.}
  \item[p321, c1] \it{Depicted as arrows, typed relations make up diagrams obtained just by arrow chaining}
\end{itemize}

\item Mention the use of $\rightharpoonup$ for finite map which is immediately understood by all that have worked with JNO, i.e. provides a piece of lingua franca for all JNO alumni.
\item Mention that the terminology + taxonomy of relations is equally recognized, but that it contains remarkably little notation. Ask the question: why? and: would it be possible to improve on this taxonomy by introducing notation?
\item Plan of the paper.
\end{itemize}

\section{Taxonomy of relations}

\begin{itemize}
\item Recap relevant stuff
\item Figure with original taxonomy
\item Observation: terminology does not express duality
\item Observation: terminology does not reflect the combination effects
\item Observation: constraints (e.g. entire, simple) are expressed as extra properties of relations (e.g. $id \in ker(R)$, $img(R) \in id$, respectively)
\item Notation is needed
\end{itemize}

\begin{tikzpicture}[description/.style={fill=white,inner sep=2pt}]
    \matrix (m) [matrix of math nodes, nodes in empty cells]
    {
                &                &           & binary\ relation &            &             & \\
      injective &                & entire    &           & simple     &             & surjective \\
                & representation &           & function  &            & abstraction & \\
                &                & injection &           & surjection &             & \\
                &                &           & bijection &            &             & \\
    };

    \path[-,font=\scriptsize]
    (m-1-4)  edge        (m-2-1)
             edge        (m-2-3)
             edge        (m-2-5)
             edge        (m-2-7)
    (m-2-1)  edge        (m-3-2)
    (m-2-3)  edge        (m-3-2)
             edge        (m-3-4)
    (m-2-5)  edge        (m-3-4)
             edge        (m-3-6)
    (m-2-7)  edge        (m-3-6)
    (m-3-2)  edge        (m-4-3)
    (m-3-4)  edge        (m-4-3)
             edge        (m-4-5)
    (m-3-6)  edge        (m-4-5)
    (m-4-3)  edge        (m-5-4)
    (m-4-5)  edge        (m-5-4)
    ;
\end{tikzpicture}

\section{Forks and handles}

\begin{itemize}
\item Proposal to add forks and handles, including the intuition behind them
\item Figure of the new taxonomy
\item Observation: Duality is expressed in notation
\item Observation: Combination effects are apparent
\item Observation: Incompleteness and lack of symmetry become evident, we can now fill in the gaps.
\end{itemize}

\begin{tikzpicture}[description/.style={fill=white,inner sep=2pt}]
    \matrix (m) [matrix of math nodes, row sep=3em,
    column sep=1em, text height=1.5ex, text depth=0.25ex, nodes in empty cells]
    {
      &   &   &   &   &   & B &   & A &   &   &   &   &   &   \\
    B &   & A &   & B &   & A &   & B &   & A &   & B &   & A \\
      &   & B &   & A &   & B &   & A &   & B &   & A &   &   \\
      &   &   &   & B &   & A &   & B &   & A &   &   &   &   \\
      &   &   &   &   &   & B &   & A &   &   &   &   &   &   \\
    };

    \path[-,font=\scriptsize]
    (m-1-7)  edge        (m-1-9)
    (m-1-8)  edge        (m-2-2)
             edge        (m-2-6)
             edge        (m-2-10)
             edge        (m-2-14)
    (m-2-1)  edge[>-]    (m-2-3)
    (m-2-2)  edge        (m-3-4)
    (m-2-5)  edge[-|]    (m-2-7)
    (m-2-6)  edge        (m-3-4)
             edge        (m-3-8)
    (m-2-9)  edge[-<]    (m-2-11)
    (m-2-10) edge        (m-3-8)
             edge        (m-3-12)
    (m-2-13) edge[|-]    (m-2-15)
    (m-2-14) edge        (m-3-12)
    (m-3-3)  edge[>-|]   (m-3-5)
    (m-3-4)  edge        (m-4-6)
    (m-3-7)  edge[-<|]   (m-3-9)
    (m-3-8)  edge        (m-4-6)
             edge        (m-4-10)
    (m-3-12) edge        (m-4-10)
    (m-3-11) edge[|-<]   (m-3-13)
    (m-4-5)  edge[>-|]   (m-4-7)
    (m-4-6)  edge        (m-5-8)
    (m-4-9)  edge[|-<|]  (m-4-11)
    (m-4-10) edge        (m-5-8)
    (m-5-7)  edge[|>-<|] (m-5-9)
    ;
\end{tikzpicture}

\section{Composing relations}

\begin{itemize}
\item We have seen that the new notation perfectly captures how the inverse operation of relations impacts the various constraints.
\item How about another basic operation on relations, i.e. composition?
\item Four basic composition properties follow immediately from the notation.
\item Others are special cases.
\end{itemize}

\section{Lifting into relators}

\begin{itemize}
\item Another basic operator: lifting into relator
\item ...
\end{itemize}

\section{Data refinement}

\begin{itemize}
\item Recall data refinement (representation and abstraction and the laws that govern them)
\item How can this be expressed with our new notation?
\end{itemize}

\section{Conclusion}

\bibliographystyle{splncs03}
\bibliography{paper}

\end{document}
