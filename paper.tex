\RequirePackage{fixltx2e}
\documentclass[runningheads,a4paper]{llncs}

\usepackage[utf8]{inputenc}

\usepackage{tikz}
\usetikzlibrary{arrows,matrix}
\usetikzlibrary{backgrounds}

\usepackage[american]{babel}

\usepackage{graphicx}

%extended enumerate, such as \begin{compactenum}
\usepackage{paralist}

%put figures inside a text
%\usepackage{picins}
%use
%\piccaptioninside
%\piccaption{...}
%\parpic[r]{\includegraphics ...}
%Text...

%Sorts the citations in the brackets
%\usepackage{cite}

%for easy quotations: \enquote{text}
\usepackage{csquotes}

\usepackage[T1]{fontenc}

%better font, similar to the default springer font
\usepackage{lmodern}
%if more space is needed, exchange lmodern by mathptmx
%\usepackage{mathptmx}

%enable margin kerning
\usepackage{microtype}

%for demonstration purposes only
\usepackage[math]{blindtext}

%unobstrusive usage of hyperref
\ifnum\pdfoutput>0
\usepackage[
%pdfauthor={},
%pdfsubject={},
%pdftitle={},
%pdfkeywords={},
bookmarks=false,
breaklinks=true,
colorlinks=true,
linkcolor=black,
citecolor=black,
urlcolor=black,
%pdfstartpage=19,
pdfpagelayout=SinglePage
]{hyperref}
%enables correct jumping to figures when referencing
\usepackage[all]{hypcap}
\else
\usepackage{hyperref}
\fi

\usepackage[capitalise,nameinlink]{cleveref}
%Nice formats for \cref
\crefname{section}{Sect.}{Sect.}
\Crefname{section}{Section}{Sections}
\crefname{figure}{Fig.}{Fig.}
\Crefname{figure}{Figure}{Figures}

\usepackage{xspace}
%\newcommand{\eg}{e.\,g.\xspace}
%\newcommand{\ie}{i.\,e.\xspace}
\newcommand{\eg}{e.\,g.,\ }
\newcommand{\ie}{i.\,e.,\ }

% correct bad hyphenation here
\hyphenation{op-tical net-works semi-conduc-tor}

\begin{document}

%Works on MiKTeX only
%hint by http://goemonx.blogspot.de/2012/01/pdflatex-ligaturen-und-copynpaste.html
%This allows a copy'n'paste of the text from the paper
\input glyphtounicode.tex
\pdfgentounicode=1

%\title{Paper Title}
\title{Putting Thought into Notation}
%\title{Relations with Forks and Handles}
%If Title is too long, use \titlerunning
%\titlerunning{Short Title}

%Single insitute
%\author{Firstname Lastname \and Firstname Lastname}
%If there are too many authors, use \authorrunning
%\authorrunning{First Author et al.}
%\institute{...}

%Multiple insitutes
%Currently disabled
%
%\iffalse
%Multiple institutes are typeset as follows:
\author{Joost Visser\inst{1} \and Miguel Ferreira\inst{2} }
%If there are too many authors, use \authorrunning
%\authorrunning{First Author et al.}

\institute{
Software Improvement Group\\
Amsterdam\\
\email{j.visser@sig.eu}\and
Schuberg Philis\\
Schiphol-Rijk\\
\email{mferreira@schubergphilis.com}
}
%\fi

\maketitle

\begin{abstract}
The work of José Nuno Oliveira is elegant. He strives relentlessly for simplicity, conciseness, symmetry, and generality. This quest for elegance finds expression in his efforts to carefully choose or craft notation. Writing things down differently allows thinking about them more clearly and reasoning about them more effectively. In this paper, we revisit a cornerstone of José Nuno’s work, the relational calculus, and examine how a novel notation for relations could bring out their basic properties more clearly. Down with arrows and harpoons that hinder our understanding. Let’s fit handles and forks onto relations to disclaim partiality and confusion.
\end{abstract}

\keywords{...}

%%%%%%%%%%%%%%%%%%%%%%%%%%%%%%%%%%%%%%%%%%%%%%%%%%%%%%%%%%%%%%%%%%%%%%%%%%%%%%%
\section{Introduction}\label{sec:intro}
%%%%%%%%%%%%%%%%%%%%%%%%%%%%%%%%%%%%%%%%%%%%%%%%%%%%%%%%%%%%%%%%%%%%%%%%%%%%%%%

\begin{tikzpicture}[description/.style={fill=white,inner sep=2pt}]
    \matrix (m) [matrix of math nodes, row sep=3em,
    column sep=1em, text height=1.5ex, text depth=0.25ex, nodes in empty cells]
    {
      &   &   &   &   &   & B &   & A &   &   &   &   &   &   \\
    B &   & A &   & B &   & A &   & B &   & A &   & B &   & A \\
      &   & B &   & A &   & B &   & A &   & B &   & A &   &   \\
      &   &   &   & B &   & A &   & B &   & A &   &   &   &   \\
      &   &   &   &   &   & B &   & A &   &   &   &   &   &   \\
    };
    %\draw[double,double distance=5pt] (m-1-1) – (m-1-3);
       \path[-,font=\scriptsize]
    (m-1-7)  edge        (m-1-9)
    (m-1-8)  edge        (m-2-2)
             edge        (m-2-6)
             edge        (m-2-10)
             edge        (m-2-14)
    (m-2-1)  edge[>-]    (m-2-3)
    (m-2-2)  edge        (m-3-4)
    (m-2-5)  edge[-|]    (m-2-7)
    (m-2-6)  edge        (m-3-4)
             edge        (m-3-8)
    (m-2-9)  edge[-<]    (m-2-11)
    (m-2-10) edge        (m-3-8)
             edge        (m-3-12)
    (m-2-13) edge[|-]    (m-2-15)
    (m-2-14) edge        (m-3-12)
    (m-3-3)  edge[>-|]   (m-3-5)
    (m-3-4)  edge        (m-4-6)
    (m-3-7)  edge[-<|]   (m-3-9)
    (m-3-8)  edge        (m-4-6)
             edge        (m-4-10)
    (m-3-12) edge        (m-4-10)
    (m-3-11) edge[|-<]   (m-3-13)
    (m-4-5)  edge[>-|]   (m-4-7)
    (m-4-6)  edge        (m-5-8)
    (m-4-9)  edge[|-<|]  (m-4-11)
    (m-4-10) edge        (m-5-8)
    (m-5-7)  edge[|>-<|] (m-5-9)
    ;
\end{tikzpicture}
\end{document}
